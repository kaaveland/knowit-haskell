\documentclass{beamer}

\usepackage[utf8]{inputenc}
\usepackage[T1]{fontenc}
\usepackage{listings}
\usetheme{Szeged}

\lstset{
  language=Haskell,
  basicstyle=\footnotesize,
  numbers=left,
  frame=single,
  showspaces=false,
  showtabs=false,
  showstringspaces=false,
  numberstyle=\tiny
}
  
\begin{document}

\begin{frame}{Agenda}
  \begin{itemize}
    \item Kjapt om syntax med eksempler og ghci-tour mm
    \item Lite, men ikke-trivielt eksempel (factor)
    \item Lite, men ikke-trivielt eksempen (calc)
  \end{itemize}
\end{frame}

\begin{frame}{Hello, World}
\lstinputlisting{hw.hs}
\lstinputlisting[language=bash]{run.sh}
\end{frame}

\begin{frame}{ghci}
  Note to self: kjapp demo (ting som er forskjellig i ghci)
  \begin{description}[align=left]
    \item[Typen til x] :t x
    \item[Informasjon om ting] :i ting
    \item[Last inn fil.hs] :l fil
    \item[Importer modul] :m Data.List Data.Map
    \item[Hjelp] :?
  \end{description}
  Kraftige greier, kan brukes til debugging mm om man vet hvordan.
\end{frame}

\begin{frame}{Kjapt om type-deklarasjoner og partials}
\lstinputlisting{typer.hs}
\end{frame}

\begin{frame}{Partials og typer del 2}
\lstinputlisting{partials.hs}  
\end{frame}


\begin{frame}{Egendefinerte typer og pattern-matching}
\lstinputlisting{egen.hs}  
\end{frame}


\begin{frame}{Mer om egendefinerte typer}
\lstinputlisting{egen2.hs}  
\end{frame}


\begin{frame}{factor}

factor [numbers]...
Print  the  prime  factors of each specified integer NUMBER.
If none are specified on the command line, read them from standard input.

\end{frame}


\begin{frame}{calc - reverse polish notation calculator}
\lstinputlisting{calc.sh}
\end{frame}

\end{document}